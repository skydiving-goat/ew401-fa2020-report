\documentclass[10pt]{article}

\usepackage{wrcecapstone}
\usepackage[utf8]{inputenc}
\hypersetup{
  pdfauthor={Lindsay Donaldson, Joe Sweeney, Lauren Modica, and Daniel Tarnawski},
  pdftitle={Skyfall},
  pdfsubject={weapons, robotics, and control engineering},
  pdfkeywords={data logging, skydiving, biomechanics}
}
\coursenumber{EW401}
\student{MIDN 1/C L.~Donaldson, MIDN 1/C J.~Sweeney,\\MIDN 1/C L.~Modica, and MIDN 1/C D.~Tarnawski}
\advisor{Assistant Professor D. Evangelista}

\title{Skyfall}
\author{MIDN 1/C L.~Donaldson, J.~Sweeney, L.~Modica, and D.~Tarnawski}
\date{\printdate{12/5/2020}}

\begin{document}
\maketitlepage
\cleardoublepage
\tableofcontents
%\listoffigures
%\listoftables

%first page
\clearpage
\maketitle

\begin{abstract}
There has become a need for a data collector in the NSW community, specifically to aid in more efficient instruction during freefall training. This semester we have tested multiple sensors to determine the most appropriate for calculating data that represents the performance and safety of a skydiver. We have tested various sensors to include the flex sensor, magnetic sensors, pneumatic sensors, computer vision, and IMU sensor. With insight from our NSW customer and AFF Instructor, we have determined safety is an utmost priority, with accuracy being a close second. We must operate within the stringent limitations of the USPA while integrating a commercial device called FlySight. We will be implementing a gyro sensor and computer vision to accurately collect the appropriate data in freefall and under canopy. Our decision to steer our focus towards the computer vision and the IMU sensors came after failed attempts to research the other sensors. Once we narrowed it down to these two sensors we made efforts to perform multiple tests. We directed our attention to computer vision because of our success in its implementation along with lack of success with the IMU. Moving forward, continue to try and implement the IMU in order to compare it against the computer vision. Additionally, we will have to create the equipment to safely and comfortably attach the sensors to the skydiver. This will require multiple tests, on the ground and in the wind tunnel, before it is taken in the air. By the end of next semester we hope to have completed multiple tests with this equipment so we can present it to a skydive instructor who can provide feedback on its usefulness in reality.
\end{abstract}

\section{Introduction}
\subsection{1.1. Customer Interview}
For this project, there are two primary customers. Mike Shultz was the first customer consulted. MIke Shultz is an AFF instructor, commercial pilot and a coach for the United States Naval Academy’s Parachute team. He aided this project by capitalizing on the sensors already on each diver. The audible, which reminds the diver of altitude, and the Cyprus, which can deploy the reserve canopy if the diver gets too low in height without pulling for the main, are both important devices every skydiver uses as a safety measure. They both have valuable sensors beneficial to this project.

CAPT (ret) Robert T. Herbert is a distinguished Military Professor of Ethics at USNA. He is a 1983 graduate of Davidson College, Master of Arts in National Security Affairs from NPS, Master of Science in National Security Studies from National War College, Ph.D. in International Relations and Political Theory from University of Virginia. He served two tours as SEAL Platoon Commander, Operations Officer of SEAL Team EIGHT, Executive Officer of SEAL Delivery Vehicle Team TWO, and Naval Special Warfare Task Unit Commander at SEAL Team TWO. His main input for the project was finding a way to collect data while keeping the diver safe.

POV: Skydivers need to safely track their performance to better their skill.

\subsection{1.2. Additional Background Research}
The majority of the background research capitalized on the importance for safety. 

Acknowledging the thought process of a safe skydiver was the first step in our background research. The observations of four divers were evaluated in the article ``The Situated Management of Safety during Risky Sport: Learning from Skydivers’ Courses of Experience'' [3]. This article aimed to determine ways to manage the risks of skydiving. The experiment took four experienced skydivers and took data through cameras and through verbal observations. The safety of diving comes down to the safety procedures demonstrated by each individual diver. The article compromised a sequence of 7 steps divers go through to minimize risk. ``They were labeled `To check the material during preparation,' `To feel prepared and safe for the jump as the plane gains altitude,' `To use the time of freefall,' `To deploy the parachute,' `To fly safely,' `To ensure a safe landing' and `To organize the structured packing of the parachute' '' [2]. The experiment observed each diver and wrote down their thoroughness to compare with each other. The results concluded that these divers had dynamic practical knowledge about skydiving safety which kept their performance level high and their risk level low. The article emphasized the need for divers to not accept the risk, but rather find embody safety check-lists to manage the risk. The graph below demonstrates the safe landing sequence observations of the divers.

Second, we consider the tracking aspect of the project. One article talks about an integrated safety/training system that combines sensors, hardware, and software technologies. The goal of this research was to improve the safety of skydiving. The intended user of this technology would not be a casual skydiver, but more of an experienced jumper looking to improve their craft. Similar to our project, the system is beneficial post-jump so the skydiver can analyze various metrics to help improve efficiency. The project gives 3-D maps and accident analysis. This type of information would be beneficial to NSW as they attempt to track students who may be unsafe. The product also has a virtual reality aspect to help a student before his/her jump. It uses GPS data in-flight to track the flight and provide information 5 minutes upon landing. This helps our project better understand the types of data we are looking to include in our tracking device. Also, the device can track tandems well which are helpful to our customer as students become more proficient they may take part in tandem jumps. This system was very thorough and did not leave out many metrics when it came to in-flight tracking, however we would like to add some more specific metrics that relate to toggle rate, pitch, roll, and yaw. The use of GPS will help us as we are still looking for ways to accurately measure jumps. GPS could end up being something we include in our final system.

Lastly, we observed a GNSS receiver, or Global Navigation Satellite System, is a system that can provide accurate GIS data collection [2]. The goal of this system was to prove that the GNSS receiver could provide accurate data in a highly dynamic environment, skydiving. The team wanted to gather attitude and positioning data. The skydiver, Levson, was instructed to perform complex skydiving maneuvers to test how accurate the positioning of the receiver really was. Based on each maneuver, the GNSS was able to collect various data, as seen in the table below. The data included is time, latitude, longitude, height, azimuth, roll, pitch, and acceleration in each of the three planes. This is very helpful for our project as we intend to use each of the data fields mentioned above in our tracking system. The article does not contain too much information about the exact engineering method used to get each piece of data, but seeing each field is helpful for us to understand the importance and difficulty of each metric. Also, it is very helpful to instructors to understand each metric when debriefing a jump. There are no shortcomings of this project, however we must do more in-depth research to fully understand the tracking system.




\section{Problem Statement}
\subsection{2.1. Problem Statement}
Our project will provide an enhanced data collector to be used in the freefall training of NSW personnel. It will be a compact device or a series of compact devices placed on a skydiver to collect data about the skydiver’s performance during freefall and canopy flight. This information will be collected for instructors to better debrief their students. We will develop a device adapted from a commercial product, called FlySight, that is already on the market in order to increase the amount and type of information collected including toggle movement, canopy reaction, and movement about all axis. An additional goal would be to develop a computer program that can organize all of that information and report a simulation or trends to the instructor.

\subsection{2.2. Functions}
Our project will be able to sense, collect, log, analyze. There are different functional phases for our project. Phase one is to sense and collect toggle inputs and collect gyro data from the skydiver with different sensors. Phase two is to be able to integrate FlySight with the data collected from phase one. We would accomplish this in one of three ways. 

Option number one would be a series connection where the toggle sensors and gyro are connected directly to the FlySight which would output all of the data collected in the same format. Option two would be a parallel connection where the toggle input sensor and gyro information funnel alongside FlySight data. In this case the data from both sources is in a different same format when entering the logger. Option three would also be a parallel connection where the data from both sources is in the same format when entering the logger. Phase two of the project’s functionality would be to log and analyze data. Finally, phase three, which may be the goal of a follow-on project, would be to create a user-friendly display for instructors to view the performance of the skydiver.

\subsection{2.3. Constraints}
There are four main constraints to our project. The first constraint is that the design must not impede the performance of the skydiver. This means that the skydiver should not notice the additional equipment on their person. The second constraint is that this project should be safe and not change the flight dynamics of the skydiver or parachute. It can be considered safe when the design follows USPA guidance and is approved by a USPA Master Rigger. The third constraint is that we will incorporate the FlySight to provide simple measurements such as speed and glide ratio. The final constraint is that it will follow USPA guidelines for students and inexperienced jumpers should it be used by those individuals.

\subsection{2.4. Objectives, Pairwise Comparison Chart, and Weightings}
The project should be safe, effective and accurate. It should be safe so that it does not interfere with the skydive and it should be lightweight and compact. It should be effective so that clear data is presented to the instructor, the equipment is reliable for the instructor, and includes all three inputs from the toggle sensor, gyro and FlySight. It should be accurate so that it properly displays a students skydive.

Table 1 Piecewise Comparison Chart

\subsection{2.5. Metrics}
The go- or no-go decision for a design based on the safety objective would be if it follows USPA guidelines. Should it follow USPA guidelines it is then a determination of the skydiver if it is comfortable to wear. The effectiveness of the design is based on the number of sensors that are consistently able to provide relevant data. As the number of sensors that provides relevant information increases so does the effectiveness. The accuracy of the design is based on how precise the toggle movement can be tracked. The more precise this is the higher the accuracy of the design.

Table 2 Cooper-Harper Rating Scale for Safety [Metric 1]

Table 3 Effectiveness [Metric 2]

Table 4 Accuracy [Metric 3]

\section{3. Related Work}
One of the main products we are basing our work around is a commercial device called the FlySight [4]. The FlySight attaches to the helmet of the skydiver and collects GPS and time data. We can take advantage of these features by integrating it with the other sensors we plan to use in order to capture the entire picture of ones skydive. When it is combined with toggle and orientation data it will allow an instructor to see how a skydiver’s position has changed as a result of their actions.

Another device that will be critical to our project is a Cypress Automated Activation Device which senses the skydivers speed and altitude to deploy their parachute should the be unable to do so [1]. This is done through a barometric pressure reading and a microprocessor the triggers the parachute if the skydiver remains at a certain speed at their critical altitude. This device could be very useful because its altitude and speed recordings must be very accurate to promote safety. We can tap into this information to integrate with our other sensors in order to get the most accurate speed and altitude information.

Finally, there is the ADA-Fruit gyroscope that will help us to determine the pitch, roll and yaw of the skydiver while they are in freefall and under canopy [5]. It is compact and small, 20x50mm and only weighing 4 grams. This is a critical device for our project because we need to now axis movements in conjunction with toggle inputs in order to provide anything useful to an instructor from a student’s skydive.

All of the devices above are important. The gyroscope and the FlySight are crucial for the success and accuracy of the project as they must be integrated with the other sensors. While the automated activation device is not crucial to the development of the project, it would provide much more sensitive data, which may be necessary to provide results with the right amount of accuracy.



\section{94.Conceptual Designs}
\subsection{4.1. Concept 1: IMU}
The IMU design consists of using three bno055 gyro sensors. Two will be located on each hand of the skydiver and another will be placed on the belt. We will use the IMU on the belt as the reference and see how far the others move from that reference. This will be able to determine the amount of toggle input from the skydive while under canopy. This design was given a 6 out of 7 for safety because it would not hinder the skydiver during his/her jump. While adding any additional equipment can be concerning, the low profile of the IMU would significantly reduce any risk. We initially rated it a 4 out of 4 for effect and 2 out 4 for accuracy. With further testing we will be able to determine how accurate this design will be.

Figure 1 IMU Functional Diagram

\subsection{4.2. Concept 2: Computer Vision}
The computer vision design involves a skydiver wearing bright colored gloves that could be picked up by a camera attached to a belt worn by the skydiver. The camera will use colorThresholder via MATLAB coding in order to find the centroid of the skydiver’s hands and therefore determine the skydiver’s hand and toggle movement. This design was given a 4 out of 7 for safety because it would cause slightly more risk for the skydiver during his/her jump. Unlike the IMU, the GoPro sticks out of the skydiver’s profile which can be an increased snag hazard. We initially rated it a 4 out of 4 for effect and 3 out 4 for accuracy. With further testing we will be able to determine how accurate this design will be.

Figure 2 Computer Vision Functional Diagram

\subsection{104.3. Decision Matrix}

Table 5 Component Decision Matrix

The IMU wins out before we have conducted any skydiving test trials because of safety. It would be much safer to put a small IMU on a skydiver’s belt as opposed to a camera. The camera would be bulkier and act as more of a snag hazard for the skydiver. While these considerations do not make the computer vision an unsafe method, it would not be quite as safe as the IMU.




\section{5. Ethical Considerations}
Our main ethical concerns revolve around the safety of the skydiver. If we create a product for the skydiver to wear and assure that skydiver it is safe we need to ensure that is actually doesn’t pose a threat. We must ensure that we do thorough tests on the ground and in the wind tunnel to ensure its safety. This will give us the peace of mind that when we ask a skydiver to make a skydive with this equipment they will be safe while doing so.

One additional concern related to safety is that by creating this device we are saying that the data we collect is accurate enough for an instructor to use for evaluation and instruction. While we do not intend for this device to replace the decision making and expertise of the instructor, we must ensure that our information is accurate. If the data collected does not accurately reflect the skydive we could be encouraging bad behavior in a student skydiver.



\section{6. Engineering Standards and Specifications}
In order to wear the Go-Pro camera and the IMU during flight, we plan on creating a belt for the skydiver to wear while meeting USPA guidelines. This belt will be specially outfitted to hold the Go-Pro camera at an angle pointed towards the skydiver’s toggles and also keep the IMU in an area that will allow it to be an effective gyroscope while maintaining the safety of the entire system. Because the area above the waste on the front side of a skydiver is safe to mount the equipment, we will use a rigid belt (like a pistol belt or something similar) and velcro to mount the Go-Pro and IMU. The velcro we will use for this will be Mil. Spec. Hook and Loop Velcro that conforms to MIL-A-A-55126B.



\section{7. Preliminary Detailed Design}
\subsection{7.1. Component Selection}
The GoPro was selected as the camera of choice for the project because it is often used by skydivers in freefall and under canopy. This is because the camera is impact rated and resistant to water and dust. Additionally, the camera is able to focus and stay steady even under wind speeds of 120 mph during freefall. Finally, GoPros are small and compact which allows them to be less of a snag hazard.

The BN0 055 was chosen because it is very available and compatible with most interfaces. Additionally, there are nine degrees of freedom which will provide us with the most accurate and in depth information from a skydive.

\subsection{7.2. Parts List and Budget}

Table 6 Parts List

Table 7 Parts List (continued)

Table 8 Budget

\subsection{7.3. Mechanical Drawings}

Figure 3 IMU Mechanical Drawing

Figure 4 Computer Vision Mechanical Drawing

\subsection{7.4. Circuit Diagrams}

Figure 5 Computer Vision Circuit Diagram

Figure 6 IMU Circuit Diagram

\subsection{7.5. Prototypes}

Figure 7 Vibrant colored hand prototype

Figure 8 Binary Hand Image

\subsection{7.6. Software Structure}

%Pseudo Code:

%-Load in Image

%-Create Threshold mask

%-For each frame in the image

%-Use the mask to create a binary image

%-Determine which blob is the hand

%- determine the size of the hand in pixels

%-use the equation to determine distance from camera from size

%-store the data in workspace array

Figure 9 Code to Create Mask

Figure 10 Code to Process Image

Figure 11 Code to Process Video

\subsection{7.7. Simulations}

Figure 12 Using the data points in the graph we created an equation to correlate the size of the hand in the image to the distance from the camera.

\section{8. Proposed Work}

\subsection{8.1. Work Breakdown Structure}

%VR Display of a Students Skydive
%1.1 Gyro Implementation
%1.1.1 Gyro type selection (Daniel - 1 hour)
%1.1.2 Gyro position and attachment (Daniel - 1 hour)
%1.2 Toggle Positioning Sensor
%1.2.1 Develop and test computer vision (Daniel and Joe - 4 hours)
%1.2.2 Develop and test IMU in elevator (Lauren and Lindsay - 4 hours)
%1.2.3 Determine best application (all members - 2 hours)
%1.2.4 Sensor position and attachment (all members - 3 hours)
%1.3 Flysight Adaptation
%1.3.1 Analyze Flysight data collected (Daniel and Joe - 4 hours)
%1.3.2 Make Flysight modifications (Lauren and Lindsay - 4 hours)
%1.4 Data Collection and Analysis
%1.4.1 Modify flysight and toggle sensor data to be comparable
%(Lauren and Lindsay - 3 hours)
%1.4.2 Create a program to present mass data collection
%(Daniel and Joe - 3 hours)
%1.4.3 Analyze the data and display students performance to the instructor
%1.5 VR Display
%1.5.1 Program data to connect data to display in VR program
%1.6 Organization and Documentation
%1.6.1 Draft report (Lauren and Lindsay - 2 hours)
%1.6.2 Draft poster (Daniel and Joe - 2 hours)
%1.6.3 Final report (Lauren and Lindsay - 1 hour)
%1.6.4 Final poster (Daniela and Joe - 1 hour)

Tasks 1.1, 1.2, and 1.3 can be completed in parallel, but we cannot move on to 1.4 until the first three are completed. Step 1.4 is where 1.1, 1.2, and 1.3 are combined in order to allow for the data collection. One aspect of 1.3 that may need to be held off until 1.1 and 1.2 are complete is making any final modifications to the FlySight if we choose to connect the FlySight with our external sensors.

The biggest hurdle we have to get over initially is the direction we want to take for the toggle sensor. Once we can determine what route we are going to take we can build the application for the skydiver and begin testing it. After this decision I foresee some issues with integrating the information from the different sensors into one big data collection program that can easily represent a student skydiver’s performance.

\subsection{8.2. Timeline}

Table 9 Spring Semester Gantt Timeline

Table 10 Spring Semester Gantt Timeline (continued)

\subsection{8.3. Risk management}
The biggest risk we will face is the safety of the skydiver when the equipment is used in an actual skydive. The critical step that will come before the execution of a skydive using this equipment will be hours and hours of testing. The devices will first be tested on the ground in the Naval Academy Parachute Team’s loft. There we will have access to skydiving equipment and we can simulate the skydiving position accurately to see if the device will be comfortable and not cause any interference. Once these trials are complete we can try and test them in a wind tunnel to see if the devices are disturbed with high wind speeds. Once we are certain these tests have proved the equipment can be safe to wear on a skydive we can test it in the air. The first test jump will be executed by someone who holds no less than a C-License in order to mitigate risk.

1. Weather -- If the weather is not ideal, the tests cannot happen. To mitigate this, we will use a harness system to test our design indoors, and allow flexibility in the schedule towards the end of the project.

2. Covid -- In order to keep everyone healthy, we need to guarantee that we follow covid social- distancing guidelines. All instrumentation can also be individually worked on, and does not need a specific space to work. With this, it is possible for each member to take a piece of the system and continue to make progress without actually having to meet up.

3. Snag Hazard -- To keep the skydiver safe, it is imperative that the equipment will not catch on the lines of the chute affecting the performance of the diver. In order to prevent this, the system will actually be stitched into wearable clothing that the diver will be able to put on underneath their suit. By doing so, there is no possible way a piece of equipment can become a snag hazard. If we choose to implement a camera system, it will be mounted on the front belt area, which is a safe area in terms of being a snag hazard.

4. Restriction of movement -- The equipment cannot be overly bulky causing the diver to lose mobility during their jump as it is unsafe. To reduce this, the stitched shirt design will ensure that the system is loose enough to be worn as a piece of clothing. This also ensures that something like a wire will stay in place, so it will be impossible for it to get wrapped around the diver in any restrictive way.

5. Equipment malfunction -- Due to the harsh nature of skydiving, it is possible that some of the equipment will break, or turn loose during a dive. In order to reduce the likelihood of this, all equipment will be stitched into a wearable shirt, in order to reduce the amount of wear from movement. This will also stop any pieces from breaking loose and falling off the diver. We will also have our design to be ``one size fits all.'' This way, there will not be a malfunction due to sizing issues.

\subsection{8.4. Demonstration and Testing Plan}
For our benchmark, we plan on getting each sensor in our system to work individually. To be considered a success, the following tests must be successful per sensor:
\begin{itemize}
\item The sensor must be able to output read data in MATLAB when stationary.
\item The sensor will then be moved a known amount, and the sensor will have to accurately display the correct movement in MATLAB
\end{itemize}
We also have a computer vision branch of the project, should our initial design fail. In this design, a camera will be attached to a belt, looking up, and use computer vision to determine the position of the person’s hand relative to the camera. For our benchmark, we will take several images of a person with red gloves at different positions. If the test goes correctly, the computer will be able to determine the distance of the hand relative to the camera, which we can then use to determine the amount of toggle movement.

We will have each sensor working separately. Each sensor of the project will be displayed individually as we justify its importance and its workability. Our focus for these tests will be on accuracy because we cannot test its safety in the air.
\begin{itemize}
\item Flysight: prior data collected will be displayed to show guests what the flysight collects
\item Gyro: device will be running while guests move it around seeing the movement data displayed on tera term
\item Computer Vision: device will be running while guests wear a red glove in front of the camera while watching the accuracy of the camera’s color threshold displayed in MATLAB
\end{itemize}



\section{9. Benchtop Demonstration}
\subsection{9.1 Activities}
We will test in the bench demonstration will be computer vision. After running into trouble implementing the IMU, we will focus on using computer vision to track the toggle movements of the skydiver in flight for the bench-top demonstration. Because we are unable to currently test our computer vision in real jumps, we will perform the demonstration in Hopper Hall using a cut-out hand and a Go-Pro Hero 6 Black camera. In order to effectively ascertain whether our MATLAB color thresholding and distance code will work during jumps, we will tape the hand against a wall and take a video with the camera from different distances. After exporting the frames of the video, we will use MATLAB to track the hand to wall distance and the area of the hand and display a video of the results for our demonstration.

\subsection{9.2. Results}
For the bench-demo we focused on Color thresholding to estimate the distance of the diver’s hands to the camera. We conducted some test photos and videos in the classroom with paper hand cutouts painted neon yellow. We were able to determine the distance of the hands within half an inch averagely. For the future, we plan to make the color thresholder more precise with more tests to guarantee its ability to properly overview the diver’s performance. We also plan to use real human hands wearing brightly colored gloves rather than cut outs to conduct more accurate tests.

Figure 13 MIDN Sweeney hard at work during bench demo preparations

%References
%[1] Airtec, "User’s Guide," ​ Cypres 2 Manual , ​ Airtec GmbH & Co. KG Safety Systems,0:25, 2009. 2
%[2] Cozzens, Tracy “Taking GNSS receiver testing to new heights”, GPS World 2020 Aug.
%[3] Sara Mohamed, Vincent Favrod, Roberta Antonini Philippe, and Dennis Hauw, “The Situated Management of Safety during Risky Sport: Learning from Skydivers’ Courses of Experience”, J Sports Sci Med, 2015 May.
%[4] “The tech behind skydiving | skydive orange,” ​ Skydiveorange.com ​ , 11-Oct-2018. [Online]. Available: https://www.skydiveorange.com/2018/10/11/the-tech-behind-skydiving/. [Accessed: 03-Sep-2020].
%[5] W. Wibisono, D. N. Arifin, B. A. Pratomo, T. Ahmad, and R. M. Ijtihadie, “Falls Detection and Notification System Using Tri-axial Accelerometer and Gyroscope Sensors of a Smartphone.” 2014 April
\nocite{*}
\bibliographystyle{IEEEtran}
\bibliography{IEEEabrv,references/skyfall.bib}
\end{document}

